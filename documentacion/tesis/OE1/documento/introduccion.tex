\section{Introducción}

La importancia de conocer el contexto en el que se desarrollan los proyectos informáticos,
especialmente si son sociales (es decir, que la sociedad se ve afecta por el uso del producto), es
esencial para el desarrollo y subsistencia de los mismos. Si se desconoce la historia, los procesos,
la política y las necesidades sociales que han impulsado al desarrollo de determinadas herramientas,
entonces se transforman en proyectos carentes de motivo, si sólo se conocen las necesidades
inmediatas que está satisfaciendo la aplicación, entonces se prevé una corta vida del proyecto.
Además, tratar de conocer el contexto histórico, social y político al que pertenece la herramienta
de software genera una visión más completa que permitirá ofrecer un producto más acabado. Por lo
anterior, se reconoce al software como una herramienta que es incapaz de escapar al contexto en el
cual se desarrolla, razón por la cual es necesario clarificar tal contexto.

Este documento adscribe al Objetivo Específico 1 del proyecto de tesis que se titula ``Sistema de
Información Geográfico Libre y Público para el Monitoreo del Estado\footnote{Por necesidades del formato del título
del proyecto de tesis, la palabra ``estado'' comienza con mayúsculas, sin embargo, se refiere a la
``situación actual'', mas no a la institución social que forma parte de la estructura
político-social de las naciones actuales.} Urbano'', por tanto se requiere conocer los motivos que
obligan a los mecanismos gubernamentales actuales y otras instituciones a tener una visión del
estado del espacio urbano, qué realidad ha hecho que en Valdivia sea necesario recabar información
más allá de fuentes empíricas y el nexo que existe con la realidad de Chile y de América Latina,
puesto que, como se verá más adelante, la historia ha permeado de manera similar en los distintos
países latinoamericanos, sobre todo en materia urbanística. Se requiere además entender cuáles son
los conceptos relativos al urbanismo y los procesos de urbanización para luego entender la necesidad
que satisfacen y la funcionalidad de los Sistemas de Información Geográficos más allá de sus
aspectos técnicos.

El siguiente documento se divide en tres secciones, la primera es una introducción a los conceptos
de urbanismo y cuáles son las herramientas actuales que determinan y ayudan a identificar los
elementos espaciales que se consideran en el estudio urbano, especialmente enfocado en Chile. La
segunda sección del documento trata sobre el desarrollo del urbanismo en América Latina en general,
sin especificar en países en concreto, salvo algunas puntuales ocasiones donde se especifican datos
sobre Chile; la importancia de esta sección se debe a que ayuda a entender cuales son las
condiciones que motivan el surgimiento de este proyecto. La tercera sección de este documento cubre
aspectos más técnicos relacionados con los Sistemas de Información Geográficos, sus orígenes, su
utilización y sus componentes.

El propósito de este documento es de carácter introductorio, por tanto, la información que contempla
puede no estar acabada en su totalidad. Por lo anterior, este documento seguirá expandiéndose en el
desarrollo del Objetivo Específico 2 del proyecto de Tesis, el cuál centrará su desarrollo sobre el
contexto chileno.
