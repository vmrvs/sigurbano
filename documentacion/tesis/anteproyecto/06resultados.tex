
\section{resultados verificables  relacionados con los objetivos específicos del
proyecto}

\begin{table}[H]
	\centering
	\begin{tabular}{|p{\textwidth}|}
        \hline
        \textbf{Objetivo Específico \#1}
        \\
        \vspace{0.5mm}
	 Estudios de conceptos clave de urbanismo y una breve introducción a los procesos de
    urbanización en América Latina; estudios generales sobre  
    los Sistemas de Información Geográficos.
        \\ \hline
        \textbf{Descripción del Resultado}
        \\
        \vspace{0.5mm}
        Documento que contenga:
        \begin{itemize}
            \item Conceptos básicos de urbanismo y recuperación
                urbana y en materia de proyectos de urbanismo a
                nivel nacional y regional.
            \item Principales tópicos de los Sistemas de Información
                Geográficos.
        \end{itemize}
        \\
        \hline
	\end{tabular}
\end{table}

\begin{table}[H]
	\centering
	\begin{tabular}{|p{\textwidth}|}
        \hline
        \textbf{Objetivo Específico \#2}
        \\
        \vspace{0.5mm}
	 Estudio del desarrollo del urbanismo en Chile, sus aspectos técnicos y legales.
    Profundización de los estudios de los Sistemas de Información Geográficos especificando
    herramientas de software presentes.
        \\ \hline
        \textbf{Descripción del Resultado}
        \\
        \vspace{0.5mm}
        Documento que contenga:
        \begin{itemize}
            \item Profundización de la historia del urbanismo en Chile, legislación y programas
              relacionados con el urbanismo.
            \item Descripción de las principales herramientas (y sus
                respectivas licencias de software) que serán
                alternativas a ser usadas en el desarrollo del
                proyecto.
        \end{itemize}
        \\
        \hline
	\end{tabular}
\end{table}

\newpage

\begin{table}[H]
	\centering
	\begin{tabular}{|p{\textwidth}|}
        \hline
        \textbf{Objetivo Específico \#3}
        \\
        \vspace{0.5mm}
        Definición de los requisitos de usuario y requisitos de
        software, estableciendo como restricciones las herramientas de
        software evaluadas en el estudio anterior.
        \\ \hline
        \textbf{Descripción del Resultado}
        \\
        \vspace{0.5mm}
        Documentos de:
        \begin{itemize}
            \item Especificación de Requisitos de Usuario.
            \item Especificación de Requisitos de Software.
        \end{itemize}
        \\
        \hline
	\end{tabular}
\end{table}

\begin{table}[H]
	\centering
	\begin{tabular}{|p{\textwidth}|}
        \hline
        \textbf{Objetivo Específico \#4}
        \\
        \vspace{0.5mm}
        Desarrollar e implementar la plataforma web colaborativa y
        escalable utilizando un marco de trabajo DSDM, basado en un
        proceso iterativo e incremental.
        \\ \hline
        \textbf{Descripción del Resultado}
        \\
        \vspace{0.5mm}
        \begin{itemize}
            \item Documentos de Análisis y Diseño del software.
            \item Documento de Especificación de Interfaz
                Humano-Computador.
            \item Prototipo finalizado, listo para ser puesto en marcha.
        \end{itemize}
        \\
        \hline
	\end{tabular}
\end{table}

\begin{table}[H]
	\centering
	\begin{tabular}{|p{\textwidth}|}
        \hline
        \textbf{Objetivo Específico \#5}
        \\
        \vspace{0.5mm}
        Desarrollar pruebas y análisis de la implementación de la
        plataforma web, verificando la satisfacción de requisitos y
        pruebas de usabilidad.
        \\ \hline
        \textbf{Descripción del Resultado}
        \\
        \vspace{0.5mm}
        Documentos sobre:
        \begin{itemize}
            \item Certificación firmada por el patrocinante en el
                que se declaran los requisitos satisfechos.
            \item Documento con el resultado de las pruebas de usabilidad con
                usuarios finales.
        \end{itemize}
        \\
        \hline
    \end{tabular}
\end{table}

